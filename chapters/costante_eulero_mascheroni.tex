% costante_eulero_mascheroni.tex
%! TeX root = ../lezione_turbo.tex
\chapter{Costante di Eulero-Mascheroni}
Durante il corso è stato dimostrato che la serie armonica $s_n=\sum^n_1\frac{1}{k}$ è asintotica a $\log(n)$.

In questa sezione si ricaverà che la loro differenza converge.

\begin{Res} Esiste un reale $\gamma\in\left( \frac{1}{2},1\right)$ tale che
	\[
		\lim\limits_{n\to+\infty}s_n-\log(n)=\gamma.
	\]
\end{Res}
\begin{proof}
	Riformuliamo $\gamma$ come un integrale e applichiamo la gamma di strumenti noti. Per farlo, riformuliamo $s_n$: si consideri il polinomio $F(x):[0,1]\to\mathbb{R}$ dato da
	\[
		F(x)=\sum^n_1\frac{x^k}{k}.
	\]

	Osserviamo che $F(1)=s_n$ e $F(0)=0$ e che la la derivata è l'$n$-esima somma parziale della successione geometrica:
	\begin{equation}
		\label{res1}
		f(x)=\sum^{n-1}_0x^k=\frac{x^n-1}{x-1}.
	\end{equation}

	Si osservi che $f$ è continua, per cui è integrabile e ammette primitiva. Allora, per il Teorema Fondamentale del Calcolo integrale e l'identità \ref{res1} si ricava che
	\[
		s_n=F(1)-F(0)=\int^1_0f(x)\:\mathrm{d}x=\int^1_0\frac{1-x^n}{1-x}\:\mathrm{d}x=\int^1_0\frac{1-(1-y)^n}{y}\:\mathrm{d}x,
	\]
	dove abbiamo sostituito $y=1-x$.

	Questo passaggio permette di passare da oggetti discreti al continuo e impiegare altri strumenti dell'analisi matematica. Poichè lavorare ulteriormente su questo integrale è difficile, conviene spezzarlo in due per linearità. In particolare,
	\begin{equation*}
		\begin{gathered}
			s_n=\mathcal{I}_n+\mathcal{J}_n \quad \text{dove} \\[0.5\baselineskip]
			\mathcal{I}_n=\int^1_0\frac{1-e^{nx}}{x}\:\mathrm{d}x \quad \text{e} \quad \mathcal{J}_n=\int^1_0\frac{e^{-nx}-(1-x)^n}{x}\:\mathrm{d}x
		\end{gathered}
	\end{equation*}

	Cominciamo con $\mathcal{J}_n$: vogliamo dimostrare che si tratta di una componente additiva infinitesima. Schematicamente, al posto di valutare l'integrale per ogni $n$ e poi prenderne il limite successionale, vogliamo trovare il limite successionale dell'integranda $f_n(x)$ per ogni $x$, e poi valutare l'integrale. Si verifica immediatamente che per $x\in\left( 0,1 \right]$ la funzione $f_n(x)$ è infinitesima al divergere di $n$. Per $x=0$ la funzione non è definita, ma può essere estesa con continuità per ogni $n$, secondo il limite
	\[
		\lim\limits_{x\to0}\frac{e^{-nx}-(1-x)^n}{x}=\lim\limits_{x\to0}\frac{o(x)}{x}=0
	\]
	calcolato sviluppando l'esponenziale con Taylor. Allora, per dei risultati sulle successioni funzionali (in particolare, la convergenza uniforme $f_n \rightrightarrows f$ e continuità di $f_n$), si ottiene che
	\[
		\lim\limits_{n\to+\infty}\int^1_0f_n(x)\:\mathrm{d}x=\int^1_0\lim\limits_{n\to+\infty}f_n(x)\:\mathrm{d}x=\int^1_00\:\mathrm{d}x=0.
	\]

	In conclusione $\lim\limits_{n\to+\infty}\mathcal{J}_n=0$ e si può dimostrare che $\mathcal{J}_n\sim\frac{1}{2n}$.

	Ora analizziamo $\mathcal{I}_n$. L'integrale è improprio, e presenterebbe un punto critico in $x=0$. In realtà, l'integranda converge per le gerarchie d'infinito e quindi è integrabile secondo Riemann su tutto l'intervallo.

	L'integrale può essere semplificato fino a raggiungere la forma "$\log x +f(x)$": si sostituisca $t=nx$ e poi si integri per parti.
	\begin{equation*}
		\begin{split}
			\mathcal{I}_n=\int^1_0\frac{1-e^{-nx}}{x}\:\mathrm{d}x&=\int^n_0\frac{1-e^{-t}}{t}\:\mathrm{d}t=\\
			&=\left[\left(1-e^{-t}\right)\log t\right]^n_0 - \int^n_0 e^{-t} \log t \:\mathrm{d}t=\\
			&=\log n-e^{-n}\log n-\int^n_0e^{-t}\log t \:\mathrm{d}t.
		\end{split}
	\end{equation*}

	Sostituendo la precedente identità nella differenza tra $s_n$ e $\log n$ otteniamo
	\[
		s_n-\log n=\mathcal{J}_n+\mathcal{I}_n-\log n=\mathcal{J}_n-e^{-n}\log n-\int^n_0 e^{-t}\log t \:\mathrm{d}t.
	\]

	Poichè $\lim_{n\to\infty} \mathcal{J}_n=0$ e $\lim_{n\to\infty} e^{-n}\log n=0$, si ricava una formula che permette di trovare la costante con precisione arbitraria:
	\[
		\gamma=\lim\limits_{n\to+\infty}(s_n-\log n)=-\int^{+\infty}_0e^{-t}\log t \:\mathrm{d}t.
	\]

	Un'osservazione interessante, infine, è che questo integrale può essere ricondotto alla Funzione Gamma.	Infatti, la derivata $\Gamma'(x)$, calcolata applicando il Teorema di Lebesegue (e cioè invertendo l'ordine di derivazione e integrazione), soddisfa la relazione

	\begin{equation*}
		\begin{split}
			\Gamma'(x)&=\frac{\mathrm{d}}{\mathrm{d}x}\left( \int^{+\infty}_0e^{-t}\: t^{\,x-1}\mathrm{d}t\right) =\\&=\int^{+\infty}_0\frac{\mathrm{d}}{\mathrm{d}x}\left( e^{-t}\: t^{\,x-1}\right) \mathrm{d}t=\\&=\int^{+\infty}_0e^{-t}\: t^{\,x-1}\log t\:\mathrm{d}t
		\end{split}
	\end{equation*}

	Da cui si deduce che $\gamma=-\Gamma'(1)$, che numericamente equivale a $0,577\dots$
\end{proof}
\pagebreak